% -*- Mode: LaTeX; TeX-PDF-mode: t -*-
\documentclass{article}
\usepackage{amsmath,proof,amsthm,graphicx}
\usepackage[section]{placeins}
\newtheorem{thm}{Theorem}
\newtheorem{lemma}{Lemma}
\newtheorem{defn}{Definition}
\newtheorem{prop}{Property}
\newcommand\set[1]{\left\{#1\right\}}
\newcommand\len[1]{\left|#1\right|}
\newcommand\paren[1]{\left({#1}\right)}
\newcommand\RS[1]{\ensuremath{\text{R}_{S#1}}}
\newcommand\WS[1]{\ensuremath{\text{W}_{S#1}}}
\newcommand\W[1]{\ensuremath{w\left[#1\right]}}
\newcommand\R[1]{\ensuremath{r\left[#1\right]}}
\newcommand\data[1]{\ensuremath{d\left[#1\right]}}
\newcommand\ip{\ensuremath{ip}}
\newcommand\op{\ensuremath{op}}
\newcommand\gneg[1]{\neg\paren{#1}}

\graphicspath{{./img/}}

\begin{document}
\section{Introduction}
The ``four-slot mechanism''\footnote{Simpson. {\em Four-slot fully
    asynchronous communication mechanism}. IEE Proceedings, Vol. 137,
  Pt. E, No. 1, January 1990.} is an asynchronous communication
mechanism designed to allow one-way, memory-less communication between
a writer program and a reader program without the use of locks. This
mechanism was later generalized to multiple readers and multiple
writers.\footnote{Simpson. {\em Multireader and multiwriter
    asynchronous communication mechanisms}. IEE Proceedings, Vol. 144,
  No. 4, July 1997.}

The following sections use the formulation of the mechanism that was
developed in the second paper, a form which varies significantly from
the original, but is more easily generalized. First is shown a proof
of coherency for the single reader version. Then it is expanded and
shown that each of the steps has a linearization
point.\footnote{Herlihy and Wing. {\em Linearizability: A correctness condition for concurrent
  objects}. ACM Transactions on Programming Languages and Systems 12
  (3): 463, 1990.} Finally, a similar proof is elaborated for the
multiple-reader version of the mechanism. In the future, a
multiple-writer version will be added.

\subsection{Definitions}
\begin{defn}[Potentially conflicting steps]
  \RS1 and \WS3 are examples of operations that take time to read or
  write the data array, and therefore, could conflict with each other
  if they are both operating on the same element of the array during
  overlapping periods of time.
\end{defn}

\begin{defn}[The ``precede'' relation]
  When operation $A$ is said to precede operation $B$, it means that
  operation $A$ occurs prior to the execution of operation $B$, and
  the full effect of operation $A$ is visible and available to
  operation $B$.
\end{defn}

\begin{defn}[Interacting operations]
  If two operations, $A$ and $B$, act upon the same memory address
  then they are said to be interacting. If those operations are
  linearizable then the ordering is strict: either $A$ precedes $B$ or
  $B$ precedes $A$.
\end{defn}

\begin{defn}[Associated operations]
  The operations which set up the state variables in preparation for
  the subsequent read or write are considered to be ``associated''
  with that read or write. For example, instances of \WS2 and \WS3 are
  associated with the subsequent instance of \WS1. And instances of
  \RS1 and \RS2 are associated with the subsequent \RS3.
\end{defn}

\subsection{Proof style}

% use of "precede" relation and why it works


%%%%%%%%%%%%%%%%%%%%%%%%%%%%%%%%%%%%%%%%%%%%%%%%%
% \begin{defn}[Atomic operation, or linearizable]
%   An atomic operation, or step, appears to happen either all at once,
%   or not at all. If two atomic operations, $A$ and $B$, occur
%   simultaneously, then the outcome of the operations will appear as if
%   $A$ preceded $B$, or as if $B$ preceded $A$. The model is strict
%   consistency, meaning that the effect of any writes to a memory
%   address by the earlier operation are available to the later
%   operation.
% \end{defn}
\section{Single Reader}
\subsection{Definitions}

\begin{itemize}
  \item $r,w$ are shared bit-arrays, $\len r=\len w=2$.
  \item $\ip,\op$ are shared bit variables
  \item $\data{\cdot,\cdot}$ is the four slot array indexed by two dimensions
\end{itemize}

\begin{prop}\label{prop:linearizable}
  Steps \WS2, \WS3, \RS1, and \RS2 are all linearizable.
\end{prop}

\subsubsection{Reader}
\begin{itemize}
\item[\RS 1] $r\gets w$
\item[\RS 2] $\op\gets\neg\ip$
\item[\RS 3] read data from $\data{\op,\R{\op}}$
\end{itemize}

\subsubsection{Writer}
\begin{itemize}
\item[\WS 1] write data into $\data{\ip,\W{\ip}}$
\item[\WS 2] $\ip\gets\neg\ip$
\item[\WS 3] $\W{\ip}\gets\neg{\R{\ip}}$
\end{itemize}

\subsection{Proofs}
\subsubsection{Reader}
The following lemmas apply to the reader code and are applicable until
the reader completes \RS3.

\begin{lemma}\label{lma:ws3notbetween}
  Given an instance of \WS3, if it does not occur in the time between
  the beginning of \RS1 and the completion of \RS3, then the
  associated \WS1 is non-conflicting because $\op\neq\ip$ or it does
  not overlap with a conflicting \RS3.
\end{lemma}

\begin{figure*}[h]
  \centering
  \includegraphics{reader_ws3notbetween}
  \caption{\WS 3 does not occur between \RS 1 and completion of \RS 3}
  \label{fig:l1i1}
\end{figure*}

\begin{proof}
  Suppose there is a \WS1 which potentially conflicts with \RS3, but
  the associated \WS3 operation occurred preceding the \RS1, as shown
  in figure~\ref{fig:l1i1}. Then by transitivity, it must be the case
  that the associated \WS2 preceded \RS2, and therefore it is safe to
  say that $\op\neq\ip$ because \RS2 sets $\op\gets\neg\ip$.
\end{proof}

\begin{lemma}\label{lma:ws3between}
  Given an instance of \WS3, if it does occur in the time between the
  beginning of \RS1 and the completion of \RS3, then the associated
  \WS1 is non-conflicting because $\R{\ip}\neq\W{\ip}$, $\op\neq\ip$
  or it does not overlap with a conflicting \RS3.
\end{lemma}

\begin{figure*}[h]
  \centering
  \includegraphics{reader_ws3between}
  \caption{\WS 3 occurs at least once between \RS 1 and completion of \RS 3}
  \label{fig:l1i2}
\end{figure*}

\begin{proof}
  Suppose there is a \WS1 which potentially conflicts with \RS3, and
  the associated \WS3 is preceded by \RS1, as shown in
  figure~\ref{fig:l1i2}. Because \RS1 controls the value of $r$, any
  occurrence of \WS3 will then set $\W{\ip}\gets\neg\R{\ip}$ so that
  $\W{\ip}\neq\R{\ip}$ for the associated \WS1.
\end{proof}

% \begin{lemma}\label{lma:rint}
%   If \RS1 precedes \WS3 then either $\W{\ip}\neq\R{\ip}$ at \RS3, or
%   that \RS3 does not conflict with any \WS1.
% \end{lemma}
% \begin{proof}
%   \RS1 controls the value of $r$ and therefore, when \WS3 occurs after
%   it, then $\W{\ip}\gets\neg{\R{\ip}}$. Now consider the case where
%   \RS3 might potentially conflict with an invocation of \WS1: the step
%   \RS3 must occur after \WS3 but before \WS2. But \WS3 has ensured
%   that $\W{\ip}\neq\R{\ip}$. No matter how many times the writer runs
%   and modifies the value of \ip, it is still the case that \WS3 will
%   ensure $\W{\ip}\neq\R{\ip}$ during a potentially conflicting \RS3.
% \end{proof}

% \begin{lemma}\label{lma:ipint}
%   If \WS2 precedes \RS2 then $\op\neq\ip$ at the following \RS3.
% \end{lemma}
% \begin{proof}
%   \WS2 controls the value of \ip\ and by the premise, \WS2 must come
%   before \RS2. Therefore, when \RS2 sets $\op\gets\neg\ip$ it must be
%   the case that $\op\neq\ip$ at \RS3.
% \end{proof}

\begin{thm}[Coherency of reader]\label{thm:rcoh}
  If \RS3 may potentially conflict with \WS1, then either $\op\neq\ip$ or $\W{\ip}\neq\R{\op}$.
\end{thm}
\begin{proof}
  A potential conflict between \RS3 and \WS1 means that \WS1 does not occur strictly before \RS2.

  If the writer has not yet completed its first invocation, and \WS2 has
  not occurred at all yet, then $\op\neq\ip$ because nothing could have
  changed the value of $\ip$.

  Interacting operations \RS2 and \WS2 must occur in one order or the
  other. The same goes for interacting operations \RS1 and \WS3.
  Lemmas~\ref{lma:ws3notbetween}~and~\ref{lma:ws3between} cover the
  cases.

 % If \WS2 precedes \RS2 then $\op\neq\ip$ by
 %  lemma~\ref{lma:ipint}.

% I want to redo this, think about interaction of WS2 and RS3.
  % If \RS2 precedes \WS2 then by transitivity it must be the case that
  % \RS1 precedes \WS3. By lemma~\ref{lma:rint} it is the case that
  % $\W{\ip}\neq\R{\ip}$. Then there are two cases: either $\op\neq\ip$,
  % in which case we are done, or $\op=\ip$, in which case
  % $\W{\ip}\neq\R{\op}$.


\end{proof}

\subsubsection{Writer}

The following lemmas apply to the writer code and are applicable until
the writer completes \WS1.

\begin{lemma}\label{lma:rs1notbetween}
  Given an instance of \RS1, if it does not occur in the time between
  the beginning of \WS3 and the completion of \WS1, then the
  associated \RS3 is non-conflicting because $\W{\ip}\neq\R{\ip}$,
  $\op\neq\ip$ or it does not overlap with a conflicting \WS1.
\end{lemma}

\begin{figure*}[h]
  \centering
  \includegraphics{writer_rs1notbetween}
  \caption{\RS1 does not occur between \WS 3 and completion of \WS 1}
  \label{fig:l3i1}
\end{figure*}

\begin{proof}
  Suppose there is a \RS3 which potentially conflicts with \WS1, but
  the associated \RS1 occurred preceding the \WS3 step, as shown in
  figure~\ref{fig:l3i1}. Suppose further that $\ip=\op$ because \RS2
  preceded \WS2. Then, because \WS3 sets $\W{\ip}\gets\neg\R{\ip}$ the
  value of $\W{\ip}\neq\R{\ip}$.
  % Suppose there is a \WS1 which potentially conflicts with \RS3, but
  % the associated \WS3 operation occurred preceding the \RS1, as shown
  % in Figure~\ref{fig:l1i1}. Then by transitivity, it must be the case
  % that the associated \WS2 preceded \RS2, and therefore it is safe to
  % say that $\op\neq\ip$ because \RS2 sets $\op\gets\neg\ip$.
\end{proof}

\begin{lemma}\label{lma:rs1between}
  Given an instance of \RS1, if it does in occur in the time between
  the beginning of \WS3 and the completion of \WS1, then the
  associated \RS3 is non-conflicting because $\ip\neq\op$ or it does
  not overlap with a conflicting \WS1.
  % Given an instance of \WS3, if it does occur in the time between the
  % beginning of \RS1 and the completion of \RS3, then the associated
  % \WS1 is non-conflicting because $\op\neq\ip$
  % or it does not overlap.
\end{lemma}

\begin{figure*}[h]
  \centering
  \includegraphics{writer_rs1between}
  \caption{\RS1 occurs at least once between \WS3 and completion of \WS1, as well as a case where $\op=\ip$}
  \label{fig:l4i1}
\end{figure*}

\begin{proof}
  Suppose there is a \RS3 which potentially conflicts with \WS1, and
  the associated \RS1 is preceded by \WS3, as shown in
  figure~\ref{fig:l4i1}. By transitivity, that means \WS2 precedes
  \RS2, but then it must be the case that $\op\neq\ip$.
  % Suppose there is a \WS1 which potentially conflicts with \RS3, and
  % the associated \WS3 is preceded by \RS1, as shown in
  % figure~\ref{fig:l1i2}. Because \RS1 controls the value of $r$, any
  % occurrence of \WS3 will then set $\W{\ip}\gets\neg\R{\ip}$ so that
  % $\W{\ip}\neq\R{\ip}$ for the associated \WS1.
\end{proof}

% \begin{lemma}\label{lma:wintopip}
%   If \WS2 precedes \RS2, then $\op\neq\ip$ during the next \WS1.
% \end{lemma}
% \begin{proof}
%   \WS2 controls the value of \ip. Every subsequent \RS2 that comes
%   after \WS2 will set $\op\gets\neg\ip$. Therefore, $\op\neq\ip$ until
%   \WS2 again, but that means the next \WS1 has completed.
% \end{proof}
% \begin{lemma}\label{lma:wbetween}
%   If \WS3 precedes \RS1 and if \WS1 occurs between \WS3 and \RS1 then $\W{\ip}\neq\R{\ip}$ at that \WS1.
% \end{lemma}
% \begin{proof}
%   \WS3 sets $\W{\ip}\gets\neg\R{\ip}$ and \RS1 sets $r\gets{}w$
%   therefore, for any \WS1 that occurs between those two steps, it must
%   be the case that $\W{\ip}\neq\R{\ip}$.
% \end{proof}

\begin{thm}[Coherency of writer]\label{thm:wcoh}
  If \WS1 may potentially conflict with \RS3, then either $\op\neq\ip$ or $\W{\ip}\neq\R{\op}$.
\end{thm}
\begin{proof}
  By theorem~\ref{thm:rcoh} we know that the reader will avoid conflicting with the writer.
  Then we need to show that the writer will avoid conflicting with the reader.

  If this is the first time that the writer runs then $\ip$ has not
  yet been modified from its initial value, and the reader must run
  \RS2 $\op\gets\neg\ip$ before reading, so $\op\neq\ip$. In
  subsequent runs of the writer code, the question becomes about how
  \WS2 and \WS3 make safe choices for the subsequent \WS1.

  Interacting operations \RS1 and \WS3 must occur in one order or the
  other. Lemmas~\ref{lma:rs1notbetween}~and~\ref{lma:rs1between} cover
  the cases.

% % so what I want to know is that WS1 does not conflict with RS3 (which may be satisfied above) but also
% % that WS2 and WS3 don't cause me to step on an ongoing read operation when WS1 comes back up again.
%   So, for one case, if \WS2 precedes \RS2, then $\op\neq\ip$ during
%   the next \WS1 by lemma~\ref{lma:wintopip}. Consider, then, the
%   alternative where \RS2 precedes \WS2 and
%   $\op\overset{?}{=}\ip$. This implies that the reader may execute
%   \RS3 and \RS1 only once until a subsequent \WS2. If, under this
%   condition, \RS1 precedes \WS3, then the next \WS1 is non-conflicting
%   because \RS3 precedes \RS1. The interesting case comes when \WS3
%   precedes \RS1 and therefore, \WS1 and \RS3 are potentially
%   overlapping in time. This case is handled by
%   lemma~\ref{lma:wbetween}.
\end{proof}

\subsubsection{Both}
\begin{thm}[Coherency]
  The reader and the writer never conflict with each other.
\end{thm}
\begin{proof}
  By theorems~\ref{thm:rcoh}~and~\ref{thm:wcoh}.
\end{proof}


\section{Single Reader (expanded)}

The steps \RS1, \RS2, \WS1 and \WS2 are, in fact, composed of multiple
atomic instructions each. The following diagrams will illustrate that
each step has a ``linearization'' point, when the effect of that step
appears to take place, and that the proofs are still valid based on
those program points.

\subsection{Expanded pseudocode}
New variables $t,u$ are introduced, temporary local variables used to
hold values that are being loaded or stored from memory by atomic
operations.
\subsubsection{Reader}
\begin{itemize}
\item[\RS {1a}] $t\gets w$
\item[\RS {1b}] $r\gets t$
\item[\RS {2a}] $t\gets\ip$
\item[\RS {2b}] $\op\gets\neg t$
\item[\RS 3] read data from $\data{\op,\R{\op}}$
\end{itemize}

\subsubsection{Writer}
\begin{itemize}
\item[\WS 1] write data into $\data{\ip,\W{\ip}}$
\item[\WS {2a}] $u\gets\ip$ (see note\footnote{Modern ARM can do \WS{2a}, \WS{2b} as a single atomic operation using {\tt LDREX, STREX}})
\item[\WS {2b}] $\ip\gets\neg u$
\item[\WS {3a}] $u\gets\R{\ip}$
\item[\WS {3b}] $\W{\ip}\gets\neg u$
\end{itemize}

\subsection{Linearization points}
To claim that a program has the property of linearizability is
equivalent to the statement that all procedures within have a
``linearization point'', which is the step when the effect of that
procedure appears to take place instantaneously.

Of the steps stipulated in proposition~\ref{prop:linearizable}, here
are their linearization points: \RS{1b}, \RS{2a}, \WS{2b}, and
\WS{3a}. Below are all the diagrams, redone with the expanded
pseudocode.

\subsection{Diagrams}

\begin{figure*}[h!]
  \centering
  \includegraphics{writer_rs1between_exp}
  \caption{\RS1 occurs at least once between \WS3 and completion of \WS1, as well as a case where $\op=\ip$}
  \label{fig:l4i1exp}
\end{figure*}

\begin{figure*}[h!]
  \centering
  \includegraphics{writer_rs1notbetween_exp}
  \caption{\RS1 does not occur between \WS 3 and completion of \WS 1}
  \label{fig:l3i1exp}
\end{figure*}

\begin{figure*}[h!]
  \centering
  \includegraphics{reader_ws3notbetween_exp}
  \caption{\WS 3 does not occur between \RS 1 and completion of \RS 3}
  \label{fig:l1i1exp}
\end{figure*}

\begin{figure*}[h!]
  \centering
  \includegraphics[width=0.75\textwidth]{reader_ws3between_exp}
  \caption{\WS 3 occurs at least once between \RS 1 and completion of \RS 3}
  \label{fig:l1i2exp}
\end{figure*}

The proofs and theorems are the same as in the previous section.

\section{Multi Reader}
\renewcommand\RS[2]{\ensuremath{\text{R}^{#1}_{S#2}}}
\renewcommand\R[2]{\ensuremath{r_{#1}\left[#2\right]}}

\subsection{Definitions}
\begin{defn}[Family of readers]
  Suppose there are $n$ readers. Each reader is denoted by $R^i$ and
  its steps by \RS i j, where $0\leq i<n$. The various variables are
  defined as follows:
\end{defn}

\begin{itemize}
\item $n$ is the number of readers.
\item $\ip,\op_0,\ldots,\op_{n-1}\in\set{0,1}$
\item the number of slots $\len{d}=2n+2$
\item $\len{w}=\len{r_0}=\ldots=\len{r_{n-1}}=2$
\item $\W 0,\R 0 0,\ldots,\R {n-1} 0\in\set{0,\ldots,n-1}$
\item $\W 1,\R 0 1,\ldots,\R {n-1} 1\in\set{0,\ldots,n-1}$
\end{itemize}

Informally, the main difference here is that we have introduced a
family of variables named $\op_i$, one for each reader, and a family
of two-element arrays named $r_i$, one for each reader. In addition,
the arrays are no longer bit-arrays but rather have each element
contain values from 0 up to $n-1$. And of course, it is no longer
merely a ``four slot'' array but a $2n+2$ slot array.

\begin{defn}[Generalized negation]
  $v' = \gneg{v_0,\ldots,v_{n-2}}$ is an operation with $n-1$
  operands. The result of this operation is defined to be unequal to
  any of the operands: for all $i<n-1$, it is the case that $v'\neq
  v_i$.  If there is more than one allowable result value, the choice
  is arbitrary. Examples: suppose $n=3$ then $\gneg{0,1}=2$ and
  $\gneg{0,0}$ can be $1$ or $2$.
\end{defn}

\begin{prop}
  Steps \WS2, \WS3, \RS i 1, and \RS i 2 are all linearizable.
\end{prop}

\subsection{Generalized negation}

discussion: $1 XOR (x_1 XOR \cdots XOR x_n)$

problem of duplicates

masking method

linearizability

\subsection{Pseudocode}
\subsubsection{Reader}
\begin{itemize}
\item[\RS i 1] $r_i\gets w$
\item[\RS i 2] $\op_i\gets\neg\ip$
\item[\RS i 3] read data from $\data{\op_i,\R i {\op_i}}$
\end{itemize}

\subsubsection{Writer}
\begin{itemize}
\item[\WS 1] write data into $\data{\ip,\W{\ip}}$
\item[\WS 2] $\ip\gets\neg\ip$
\item[\WS 3] $\W{\ip}\gets\gneg{\R 0 {\ip}, \ldots, \R {n - 1} {\ip}}$
\end{itemize}


\subsection{Proofs}
\subsubsection{Reader}

The following lemmas apply to the code of reader $R^i$ and are
applicable until that reader completes \RS i 3.

\begin{lemma}\label{lma:mws3notbetween}
  Given an instance of \WS3, if it does not occur in the time between
  the beginning of \RS i 1 and the completion of \RS i 3, then the
  associated \WS1 is non-conflicting because $\ip\neq\op_i$ or it does
  not overlap with a conflicting \RS i 3.
\end{lemma}

\begin{figure*}[h]
  \centering
  \includegraphics{mreader_ws3notbetween}
  \caption{\WS 3 does not occur between \RS i 1 and completion of \RS i 3}
  \label{fig:mrws3nb}
\end{figure*}

\begin{proof}
  Suppose there is a \WS1 which potentially conflicts with \RS i 3, but
  the associated \WS3 operation occurred preceding the \RS i 1, as shown
  in figure~\ref{fig:mrws3nb}. Then by transitivity, it must be the case
  that the associated \WS2 preceded \RS i 2, and therefore it is safe to
  say that $\ip\neq\op_i$ because \RS i 2 sets $\op_i\gets\neg\ip$.
\end{proof}

\begin{lemma}\label{lma:mws3between}
  Given an instance of \WS3, if it does occur in the time between the
  beginning of \RS i 1 and the completion of \RS i 3, then the associated
  \WS1 is non-conflicting because $\W{\ip}\neq\R{i}{\ip}$, $\ip\neq\op_i$
  or it does not overlap with a conflicting \RS i 3.
\end{lemma}

\begin{figure*}[h]
  \centering
  \includegraphics{mreader_ws3between}
  \caption{\WS 3 occurs at least once between \RS i 1 and completion
    of \RS i 3, as well as a case where $\ip=\op_i$}
  \label{fig:mrws3b}
\end{figure*}

\begin{proof}
  Suppose there is a \WS1 which potentially conflicts with \RS i 3, and
  the associated \WS3 is preceded by \RS i 1, as shown in
  figure~\ref{fig:mrws3b}. Because \RS i 1 controls the value of $r_i$, any
  occurrence of \WS3 will then set $\W{\ip}\gets\neg\R{i}{\ip}$ so that
  $\W{\ip}\neq\R{i}{\ip}$ for the associated \WS1.
\end{proof}

\begin{thm}[Coherency of reader]\label{thm:mrcoh}
  If \RS i 3 may potentially conflict with \WS1, then either
  $\ip\neq\op_i$ or $\W{\ip}\neq\R i {\op}$.
\end{thm}
\begin{proof}
  A potential conflict between \RS i 3 and \WS1 means that \WS1 does not occur strictly before \RS i 2.

  If the writer has not yet completed its first invocation, and \WS2
  has not occurred at all yet, then $\ip\neq\op_i$ because nothing
  could have changed the value of $\ip$.

  Interacting operations \RS i 2 and \WS2 must occur in one order or
  the other. The same goes for interacting operations \RS i 1 and
  \WS3.  Lemmas~\ref{lma:mws3notbetween}~and~\ref{lma:mws3between}
  cover the cases.
\end{proof}

\subsubsection{Writer}

The following lemmas apply to the writer code and are applicable until
the writer completes \WS1.

\begin{lemma}\label{lma:mrs1notbetween}
  Given an instance of \RS i 1, if it does not occur in the time between
  the beginning of \WS3 and the completion of \WS1, then the
  associated \RS i 3 is non-conflicting because $\W{\ip}\neq\R{i}{\ip}$,
  $\ip\neq\op_i$ or it does not overlap with a conflicting \WS1.
\end{lemma}

\begin{figure*}[h]
  \centering
  \includegraphics{mwriter_rs1notbetween}
  \caption{\RS i 1 does not occur between \WS 3 and completion of \WS 1}
  \label{fig:mwrs1nb}
\end{figure*}

\begin{proof}
  Suppose there is a \RS i 3 which potentially conflicts with \WS1, but
  the associated \RS i 1 occurred preceding the \WS3 step, as shown in
  figure~\ref{fig:mwrs1nb}. Suppose further that $\ip=\op_i$ because \RS i 2
  preceded \WS2. Then, because \WS3 sets $\W{\ip}\gets\neg\R i {\ip}$ the
  value of $\W{\ip}\neq\R i {\ip}$.
\end{proof}

\begin{lemma}\label{lma:mrs1between}
  Given an instance of \RS i 1, if it does in occur in the time
  between the beginning of \WS3 and the completion of \WS1, then the
  associated \RS i 3 is non-conflicting because $\ip\neq\op_i$ or it
  does not overlap with a conflicting \WS1.
\end{lemma}

\begin{figure*}[h]
  \centering
  \includegraphics{mwriter_rs1between}
  \caption{\RS i 1 occurs at least once between \WS3 and completion of \WS1, as well as a case where $\op_i=\ip$}
  \label{fig:mwrs1b}
\end{figure*}

\begin{proof}
  Suppose there is a \RS i 3 which potentially conflicts with \WS1, and
  the associated \RS i 1 is preceded by \WS3, as shown in
  figure~\ref{fig:mwrs1b}. By transitivity, that means \WS2 precedes
  \RS i 2, but then it must be the case that $\ip\neq\op_i$.
\end{proof}

\begin{thm}[Coherency of writer]\label{thm:mwcoh}
  For any reader $R^i$, if \WS1 may potentially conflict with \RS i 3,
  then either $\ip\neq\op_i$ or $\W{\ip}\neq\R i {\op}$.
\end{thm}
\begin{proof}
  By theorem~\ref{thm:mrcoh} we know that the reader $R^i$ will avoid
  conflicting with the writer.  Then we need to show that the writer
  will avoid conflicting with the reader.

  If this is the first time that the writer runs then $\ip$ has not
  yet been modified from its initial value, and the reader must run
  \RS i 2 $\op_i\gets\neg\ip$ before reading, so $\ip\neq\op_i$. In
  subsequent runs of the writer code, the question becomes about how
  \WS2 and \WS3 make safe choices for the subsequent \WS1.

  Interacting operations \RS i 1 and \WS3 must occur in one order or the
  other. Lemmas~\ref{lma:mrs1notbetween}~and~\ref{lma:mrs1between} cover
  the cases.
\end{proof}

\subsubsection{Both}

\begin{thm}[Coherency of multireader mechanism]
  The reader and the writer never conflict with each other.
\end{thm}
\begin{proof}
  By theorems~\ref{thm:mrcoh}~and~\ref{thm:mwcoh}.
\end{proof}

% \begin{lemma}\label{lma:rintn}
%   If \RS i 1 precedes \WS3 then $\W{\ip}\neq\R i {\ip}$ at \RS i 3.
% \end{lemma}
% \begin{proof}
%   \RS i 1 controls the value of $r_i$ and therefore, if \WS3 occurs after
%   it, then $\W{\ip}\gets\gneg{\ldots,\R i {\ip},\ldots}$. Therefore,
%   $\W{\ip}\neq\R i {\ip}$ until \RS i 1 runs again, which would contradict the
%   premise.
% \end{proof}

% \begin{lemma}\label{lma:ipintn}
%   If \WS2 precedes \RS i 2 then $\op_i\neq\ip$ at \RS i 3.
% \end{lemma}
% \begin{proof}
%   \WS2 controls the value of \ip\ and by the premise, \WS2 must come
%   before \RS i 2. Therefore, when \RS i 2 sets $\op_i\gets\neg\ip$ it must be
%   the case that $\op_i\neq\ip$ at \RS i 3.
% \end{proof}

% \begin{thm}[Coherency of reader]
%   If \RS i 3 may potentially conflict with \WS1, then either $\op_i\neq\ip$ or $\W{\ip}\neq\R i {\op_i}$.
% \end{thm}
% \begin{proof}
%   A potential conflict between \RS i 3 and \WS1 means that \WS1 does not occur strictly before \RS i 2.

%   If the writer has not yet completed its first invocation, and \WS2
%   has not occurred at all yet, then $\op_i\neq\ip$ because nothing
%   could have changed the value of $\ip$.

%   Interacting operations \RS i 2 and \WS2 must occur in one order or the
%   other. If \WS2 precedes \RS i 2 then $\op_i\neq\ip$ by
%   lemma~\ref{lma:ipintn}.

%   If \RS i 2 precedes \WS2 then by transitivity it must be the case that
%   \RS i 1 precedes \WS3. By lemma~\ref{lma:rintn} it is the case that
%   $\W{\ip}\neq\R i {\ip}$. Then there are two cases: either $\op_i\neq\ip$,
%   in which case we are done, or $\op_i=\ip$, in which case
%   $\W{\ip}\neq\R i {\op_i}$.
% \end{proof}

% \begin{thm}[Coherency of writer]
%   For any \RS i 3 such that \WS1 may potentially conflict with that
%   \RS i 3, either $\op_i\neq\ip$ or $\W{\ip}\neq\R i {\op_i}$.
% \end{thm}
% \begin{proof}
%   in progress
% \end{proof}

\end{document}
